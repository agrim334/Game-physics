\documentclass{beamer}
\usetheme{Warsaw}
\usepackage{nhtvslides}
\usepackage{graphicx}
\usepackage{listings}
\lstset{language=CAML,
basicstyle=\ttfamily\footnotesize,
frame=shadowbox,
breaklines=true}
\usepackage[utf8]{inputenc}

\title{Syllabus}

\author{Dr. Giuseppe Maggiore}

\institute{NHTV University of Applied Sciences \\ 
Breda, Netherlands}

\date{}

\begin{document}
\maketitle

\begin{slide}{Syllabus}{List of topics}{
\item \textbf{Topic 1 - basic concepts from physics}: translational and rotational Newtonian physics, numerical integration, equations of motion for a system of bodies
\item \textbf{Topic 2 - narrow phase of collision detection}: separating axis, collision response
\item \textbf{Topic 3 - broad phase of collision detection}: axis aligned bounding boxes, bounding spheres, etc.
\item \textbf{Topic 4 - simultaneous resolution of multiple constraints}: constraints as a system of equations, the Gauss-Seidel method
\item \textbf{Topic 5 - force computation}: ballistic forces (Magnusson, friction, gravity), car forces, plane forces, etc. 
}\end{slide}

\begin{slide}{Syllabus}{List of topics}{
\item \textbf{Optional topic 1 - preprocessing of models for collision detection} BSP for faster collision detection
\item \textbf{Optional topic 2 - preprocessing of generic models} calculating the \textit{inertia tensor} of arbitrary polytopes
}\end{slide}

\begin{slide}{Syllabus}{List of materials}{
\item The book \textit{Game Physics - Second Edition}, by David Eberly
\item The book \textit{Physics for game programmers}, by Grant Palmer
\item The paper \textit{Iterative Dynamics with Temporal Coherence}, by Erin Catto 
\item The tutorial \textit{Car physics for games}, by Marco Monster
\item The Siggraph '97 course notes \textit{An Introduction to Physically Based Modeling: Rigid Body Simulation I - Unconstrained Rigid Body Dynamics} by David Baraff 
}\end{slide}

\begin{slide}{Syllabus}{Assignments}{
\item Due
\begin{itemize}
\item The end of the week after presentation in class (possible time bonus)
\item All together at the end of the course
\end{itemize}
\item Group work for coding (max three students)
\item Individual work for the report (what will actually be graded)
}\end{slide}

\begin{slide}{Syllabus}{List of assignments}{
\item Build a basic kinematic simulator with RK2 or RK4 (20\%)
\item SAT/contact manifold computation (at least for OBBs, better for arbitrary meshes) (20\%)
\item Collision culling with bounding spheres, AABBs, and bins (20\%)
\item Collision response (20\%)
\item Forces for domain-specific scenarios (20\%)
}\end{slide}

\begin{frame}{That's it}
\center
\fontsize{18pt}{7.2}\selectfont
Thank you!
\end{frame}

\end{document}


\begin{slide}{SECTION}{SLIDE}{
\item i
}\end{slide}

\begin{frame}[fragile]{SLIDE}
\begin{lstlisting}
CODE
\end{lstlisting}
\end{frame}

\begin{frame}{SLIDE}
\center
%\includegraphics[height=5cm]{Pics/recursive_multiplier.png}
\end{frame}
